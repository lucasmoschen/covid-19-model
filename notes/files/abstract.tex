{\bf Resumo}. A doença COVID-19 causada pelo vírus SARS-CoV-2 espalhou-se rapidamente pelo mundo desde o início de 2020 e entender sua dinâmica na população é importante para tomar medidas que contenham a disseminação. 
Nesse relatório, o modelo epidemiológico SEIAQR para a COVID-19 é considerado para compreender o início da epidemia no município do Rio de Janeiro e, em especial, a taxa de subnotificação, isto é, a proporção de indivíduos infectados que não foram registrados pelo sistema \cite{aronna2021}. 
As curvas de casos confirmados e óbitos foram ajustadas aos dados reais da cidade usando o método de mínimos quadrados ponderados dos erros. 
A transmissibilidade e a mortalidade da doença são aproximadas por B-splines cujos parâmetros também foram estimados. 
Foi analisada a identificabilidade estrutural do modelo e a identificabilidade prática do ajuste para verificar a viabilidade das estimações. Utilizamos o método Bootstrap para quantificar a incerteza sobre as estimativas. Para o período março-julho de 2020, obtemos a estimativa pontual $0.9$ para a subnotificação com intervalo de confiança 95\% (0.85, 0.93).

\noindent
{\bf Palavras-chave}. Modelo COVID-19, estimativa de parâmetros,
identificabilidade, mínimos quadrados, B-splines, Bootstrap.