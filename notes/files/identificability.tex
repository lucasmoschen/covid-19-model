O primeiro passo após a concepção do modelo é analisar sua {\em identificabilidade}, porque a estimação pode variar dependendo de quando o problema é ou não bem-posto. 
De forma geral, essa análise identifica se os parâmetros desconhecidos podem ser estimados de forma única \cite{audoly2001, sarcomani2019}. 
Existem duas formas de fazer essa análise: a forma estrutural (a priori) e a forma prática (a posteriori). 
A primeira é uma propriedade teórica que reside na própria estrutura do modelo. Os parâmetros são {\em estruturalmente identificáveis} se eles podem ser unicamente estimados a partir do experimento desenvolvido e são chamados {\em localmente estruturalmente identificáveis} se essa característica é verificada em torno do ponto ótimo. 
A segunda é feita após o processo de ajuste de dados, quando outros problemas podem aparecer em relação ao ruído encontrado na informação. Em particular, utilizaremos a {\em matriz de correlação} dos parâmetros.

Conforme descrito em \cite{ljung1994}, considere um sistema dinâmico
\begin{equation}
    \begin{aligned}
        \Dot{x} &= f(x(t), \theta), &x(0) = x_0 \\
        y(t) &= h(x(t), \theta)
    \end{aligned}
\end{equation}
em que $x(t) \in \R^n, y(t) \in \R^m$ e $f$ e $h$ são vetores de funções racionais da variável $x$ e $\theta \in \Theta \subset \R^p$. 
A variável $y$ é observável e no nosso modelo é indicada pelos casos confirmados e mortes. 
Quando $x(0) = x_0$, denotamos $y = \psi_{x_0}(\theta, u)$ como a observação com esse valor inicial. 
O objetivo é contar o número de soluções da equação:
\begin{equation}
     \label{identificability-equation}
     \psi_{x_0}(\theta, u) =  \psi_{x_0}(\theta^*, u).
\end{equation}
Dizemos que esse modelo é {\em globalmente identificável} na solução $\theta ^*$ se a equação \eqref{identificability-equation} tem solução única $\theta = \theta^*$, para todo tempo $s$. 
Ele será {\em localmente identificável} se essa propriedade valer em uma vizinhança de $\theta ^*$. Aplicamos um método computacional de identificabilidade estrutural através do programa DAISY \cite{bellu2007}.

\subsubsection{DAISY - Differential Algebra for Identifiability of SYstems}

DAISY é um {\it software}, com base na linguagem de programação {\it Reduce},
específico para o problema de identificabilidade em sistemas dinâmicos, com
algumas condições brandas para a utilização. 
Se o modelo possui essas condições, a codificação e manuseio são fáceis, o que é
uma vantagem quando queremos aplicar os algoritmos algébricos envolvidos. 
Tratamos todos os parâmetros do modelo indicados na Seção \ref{apresentation} como
invariantes no tempo. 
Supondo o conhecimento da curva de recuperados $R$, o problema foi considerado
globalmente identificável. 
Infelizmente esses dados não são completamente disponíveis, conforme destacado
na Seção \ref{data-analysis} e na ausência deles,  o {\it software} não conseguiu
resolver em pelo menos 3 dias, e, portanto, decidimos não continuar por esse 
caminho. 
O código pode ser encontrado no Github \cite{github}.