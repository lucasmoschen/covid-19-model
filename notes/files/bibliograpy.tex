\bibitem{audoly2001}
Audoly, S. {\it et al}. Global identifiability of nonlinear models of biological
systems.{\it IEEE Transactions on Biomedical Engineering} 48(1):55-65, 2001, DOI: 10.1109/10.900248.

\bibitem{aronna2021}
Aronna, M. S., Guglielmi, R., Moschen, L. M. A model for COVID-19 with
isolation, quarantine and testing as control measures. {\it Epidemics}. 34,
2021. DOI: 10.1016/j.epidem.2021.100437

\bibitem{rtBrasil2020}
Bastos, S. B., Cajueiro, D. O. Modeling and forecasting the early evolution of
the covid-19 pandemic in brazil. {\it Scientific Reports}, 10(1):19457, 2020.
DOI: 10.1038/s41598-020-76257-1.

\bibitem{bellu2007}
Bellu, G., Saccomani, M. P., Audoly, S., D'Angiò, L. DAISY: a new software
tool to test global identifiability of biological and physiological systems.
{\it Comput Methods Programs Biomed}. 88(1):52-61, 2007.
DOI:10.1016/j.cmpb.2007.07.002.

\bibitem{byrd1995}
Byrd, R. H., Lu P., Nocedal, J., Zhu, C. A Limited Memory Algorithm for Bound
Constrained Optimization. {\it SIAM Journal on Scientific and Statistical
Computing} 16(5): 1190-1208, 1995. DOI: 10.1137/0916069

\bibitem{recovery2020}
Byrne A. W. {\it et al}. Inferred duration of infectious period of SARS-CoV-2:
rapid scoping review and analysis of available evidence for asymptomatic and
symptomatic COVID-19 cases. {\it BMJ Open}. 10(8), 2020. DOI: 10.1136/bmjopen-2020-039856.

\bibitem{subnotificacao-folha-de-sao-paulo}
Canzian, F. Estados e municípios no país relatam subnotificação gigantesca de casos. {\it Folha de São Paulo}. \url{https://www1.folha.uol.com.br/equilibrioesaude/2020/04/estados-e-municipios-no-pais-relatam-subnotificacao-gigantesca-de-casos.shtml}, abril de 2020. 

\bibitem{cao2012}
Cao, J., Huang J. Z., Wu H. Penalized Nonlinear Least Squares Estimation of Time-Varying Parameters in Ordinary Differential Equations. 
{\it Journal of Computational and Graphical Statistics}, 21:1, 42-56, 2012.
DOI: 10.1198/jcgs.2011.10021

\bibitem{croda2020}
Croda, J. {\it et al}. COVID-19 in Brazil: advantages of a socialized unified health
system and preparation to contain cases. {\it Rev. Soc. Bras. Med. Trop.} 53,
2020. DOI: 10.1590/0037-8682-0167-2020

\bibitem{dantas2020}
Dantas, G. {\it et al}. The impact of COVID-19 partial lockdown on the air quality
of the city of Rio de Janeiro, Brazil. {\it Science of The Total Environment},
729, 2020. DOI: 10.1016/j.scitotenv.2020.139085.

\bibitem{efron1986}
Efron, B., Tibshirani, R. Bootstrap Methods for Standard Errors, Confidence
Intervals, and Other Measures of Statistical Accuracy. {\it Statist. Sci.}
1(1):54-75, 1986. DOI: 10.1214/ss/1177013815

\bibitem{sistemas-bandeiras}
Estadão. Governo do Rio cria classificação em 3 bandeiras para flexibilizar isolamento. \url{https://revistapegn.globo.com/Noticias/noticia/2020/05/pegn-governo-do-rio-cria-classificacao-em-3-bandeiras-para-flexibilizar-isolamento.html}, maio de 2020.

\bibitem{ibge-pnad}
IBGE – Instituto Brasileiro de Geografia e Estatística. Pesquisa Nacional por
Amostra de Domicílios - PNAD COVID19.
\url{https://www.ibge.gov.br/estatisticas/sociais/trabalho/27946-divulgacao-semanal-pnadcovid1.html?=&t=downloads}.
2020.

\bibitem{ibge-rio}
IBGE – Instituto Brasileiro de Geografia e Estatística. Cidades e Estados.
\url{https://www.ibge.gov.br/cidades-e-estados/rj/rio-de-janeiro.html}.
2021.

\bibitem{jarque1980}
Jarque, C. M., Bera, A. K. Efficient tests for normality, homoscedasticity and serial independence of regression residuals.
{\it Economics Letters}, 6(3): 255-259, 1980. DOI:
10.1016/0165-1765(80)90024-5

\bibitem{joshi2006}
Joshi, M., Seidel-Morgenstern, A. Kremling, A. Exploiting the bootstrap method
for quantifying parameter confidence intervals in dynamical systems. {\it
Metabolic Engineering}, 8(5): 447-455, 2006. DOI: 10.1016/j.ymben.2006.04.003

\bibitem{delta2020}
Kucirka L. M. {\it et al}. Variation in
false-negative rate of reverse transcriptase polymerase chain reactionbased
SARS-CoV-2 tests by time since exposure. {\it Ann Intern Med}. 173(4):262–7,
2020. DOI: 10.7326/M20-1495

\bibitem{latent2020}
Li R, Pei S, Chen B, {\it et al}. Substantial undocumented infection
facilitates the rapid dissemination of novel coronavirus (SARS-CoV-2). {\it Science}.
368(6490):489-493, 2020. DOI:10.1126/science.abb3221

\bibitem{liang2010}
Liang, H., Miao, H., Wu, H. Estimation of constant and time-varying dynamic parameters of HIV infection in a nonlinear differential equation model. {\it Ann. Appl. Stat. 4}, 1: 460-483, 2010. DOI 10.1214/09-AOAS290

\bibitem{ljung1978}
Ljung, G. M., Box, G. E. P. On a measure of lack of fit in time series models.
{\it Biometrika}, 65:297-303, 1978. DOI: 10.1093/biomet/65.2.297

\bibitem{ljung1994}
Ljung, L. , Glad, T. On global identifiability for arbitrary model
parametrizations. {\it Automatica}, 30: 265-276, 1994.
DOI:10.1016/0005-1098(94)90029-9.

\bibitem{rt-imperial-college}
Mellan, T., {\it et al.} Subnational analysis of the COVID-19 epidemic in
Brazil. {\it Cold Spring Harbor Laboratory Press}, 2020. DOI:
10.1101/2020.05.09.20096701. 

\bibitem{miao2011}
Miao, H., Xia, X. Perelson, A. S., Wu, H. On Identifiability of Nonlinear ODE Models and Applications in Viral Dynamics.
{\it SIAM Review}, 53(1): 3-39, 2011. DOI: 10.1137/090757009

\bibitem{github}
Moschen, L. M. Repositório Covid-19. {\it Github}. Disponível em \url{https://github.com/lucasmoschen/covid-19-model}.

\bibitem{nogrady}
Nogrady, B. What the data say about asymptomatic COVID infections. {\it
Nature}. 587: 534-535, 2020. DOI: 10.1038/d41586-020-03141-3

\bibitem{observatorio}
Observatório COVID-19 BR. R efetivo no Rio de Janeiro. Disponível em
\url{https://covid19br.github.io/municipios.html?aba=aba3&uf=RJ&mun=Rio_de_Janeiro}.
Acesso em abril de 2021. 

\bibitem{subnotification2-brazil}
Portal COVID-19 Brasil. COVID-19 BRASIL [acessado 2021 Abril]. Disponível em: \url{https://ciis.fmrp.usp.br/covid19/}

\bibitem{subnotification-brazil}
Prado, M. F. do {\it et al.} Análise de subnotificação do número de casos confirmados da COVID-19 no
Brasil. Disponível em
\url{https://drive.google.com/file/d/1_whlqZnGgvqHuWCG4-JyiL2X9WXpZAe3/view}. 

\bibitem{subnotification-rio}
Prado, M. F. do {\it et al}. Análise da subnotificação de COVID-19 no Brasil.
{\it Rev. bras. ter. intensiva [online]}. 32(2): 224-228, 2020. DOI: 10.5935/0103-507x.20200030.


\bibitem{incubation2020}
Rai, B., Shukla, A., Dwivedi, L.K. Incubation period for COVID-19: a
systematic review and meta-analysis. {\it J Public Health (Berl.)} 2021. DOI: 10.1007/s10389-021-01478-1

\bibitem{ramsey2007} 
Ramsay, J. O., Hooker, G., Campbell, D. e Cao, J. Parameter estimation for
differential equations: a generalized smoothing approach. {\it Journal of the
Royal Statistical Society: Series B (Statistical Methodology)}, 69:741--796,
2007. DOI:10.1111/j.1467-9868.2007.00610.x

\bibitem{decreto-emergencia}
Rio de Janeiro. Decreto nº 46.973, 16 de março de 2020. Reconhece a situação de emergência na saúde pública do estado do Rio de Janeiro em razão do contágio e adota medidas de enfrentamento da propagação decorrente do novo coronavírus (COVID-19); e dá outras previdências. {\bf Diário Oficial do Estado do Rio de Janeiro}, Rio de Janeiro, RJ, n. 049-A, 17 de março de 2020. Disponível em \url{https://pge.rj.gov.br/comum/code/MostrarArquivo.php?C=MTAyMjI}. 

\bibitem{decreto-mascaras}
Rio de Janeiro. Lei nº 8859, 03 de junho de 2020. Estabelece a obrigatoriedade do uso de máscaras respiratórias, no âmbito do estado do Rio de Janeiro, enquanto vigorar o estado de calamidade pública em virtude da pandemia do novo coronavírus (COVID-19). {\bf Diário Oficial do Estado do Rio de Janeiro}, Rio de Janeiro, RJ, n. 100, 04 de junho de 2020. Disponível em \url{http://www.aeerj.net.br/file/04-06-2020-leiestadomascara.pdf}. 

\bibitem{rodriguez2006}
Rodriguez-Fernandez, M., Egea, J. A., Banga, J. R. Novel metaheuristic for parameter estimation in nonlinear dynamic biological systems. {\it BMC Bioinformatics} 7:483, 2006. DOI: 10.1186/1471-2105-7-483

\bibitem{sarcomani2019}
Saccomani M. P., Thomaseth K. Calculating all multiple parameter solutions of
ODE models to avoid biological misinterpretations.{\it Mathematical
Biosciences and Engineering : MBE }, 16(6):6438-6453, 2019. DOI:
10.3934/mbe.2019322.

\bibitem{data-rio-covid}
Secretaria Municipal de Saúde (SMS), Prefeitura da Cidade do Rio de Janeiro. Dados individuais dos casos confirmados de COVID-19 no município do Rio de Janeiro. \url{https://www.arcgis.com/home/item.html?id=f314453b3a55434ea8c8e8caaa2d8db5}, março de 2021.

\bibitem{painel-rio-covid}
Secretaria Municipal de Saúde (SMS), Prefeitura da Cidade do Rio de Janeiro. Painel Rio COVID-19.\url{https://experience.arcgis.com/experience/38efc69787a346959c931568bd9e2cc4}, 2021.

\bibitem{sars-cov-2}
The 2019 nCoV Outbreak Joint Field Epidemiology Investigation Team and Q. Li.
An Out-break of NCIP (2019-nCoV) Infection in China - Wuhan, Hubei Province,
2019 -
2020.-\url{http://weekly.chinacdc.cn/en/article/id/e3c63ca9-dedb-4fb6-9c1c-d057adb77b57},
janeiro de 2020

\bibitem{scipy}
Virtanen P. {\it et al}. SciPy 1.0: Fundamental Algorithms for Scientific Computing
in Python. {\it Nature Methods}. 17(3): 261-272, 2020. 

\bibitem{oms-symptoms}
World Health Organization. Coronavirus disease (COVID-19).
\url{https://www.who.int/emergencies/diseases/novel-coronavirus-2019/question-and-answers-hub/q-a-detail/coronavirus-disease-covid-19#:~:text=symptoms},
outubro de 2020